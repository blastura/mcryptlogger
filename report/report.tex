\documentclass[a4paper, 12pt]{article}
\usepackage[swedish]{babel}
\usepackage[utf8]{inputenc}
\usepackage{verbatim}
\usepackage{fancyhdr}
\usepackage{graphicx}
\usepackage{parskip}
\usepackage{minitoc}

% Include pdf with multiple pages ex \includepdf[pages=-, nup=2x2]{filename.pdf}
\usepackage[final]{pdfpages}
% Place figures where they should be
\usepackage{float}

% Float for text
\floatstyle{ruled}
\newfloat{xml}{H}{lop}
\floatname{xml}{XML}

% vars
\def\title{mcryptlogger}
\def\preTitle{Laboration 3}
\def\kurs{Systemprogrammering, VT-09}

\def\namn{Anton Johansson}
\def\mail{dit06ajn@cs.umu.se}
\def\pathtocode{$\sim$dit06ajn/edu/5DV004/lab3}

\def\handledareEtt{Johan Eliasson, johane@cs.umu.se}
\def\handledareTva{Johan Granberg, johang@cs.umu.se}

\def\inst{datavetenskap}
\def\dokumentTyp{Laborationsrapport}

\begin{document}
\begin{titlepage}
  \thispagestyle{empty}
  \begin{small}
    \begin{tabular}{@{}p{\textwidth}@{}}
      UMEÅ UNIVERSITET \hfill \today \\
      Institutionen för \inst \\
      \dokumentTyp \\
    \end{tabular}
  \end{small}
  \vspace{10mm}
  \begin{center}
    \LARGE{\preTitle} \\
    \huge{\textbf{\kurs}} \\
    \vspace{10mm}
    \LARGE{\title} \\
    \vspace{15mm}
    \begin{large}
        \namn, \mail \\
        \texttt{\pathtocode}
    \end{large}
    \vfill
    \large{\textbf{Handledare}}\\
    \mbox{\large{\handledareEtt}}
    \mbox{\large{\handledareTva}}
  \end{center}
\end{titlepage}

\newpage
\mbox{}
\vspace{70mm}
\begin{center}
% Dedication goes here
\end{center}
\thispagestyle{empty}
\newpage

\pagestyle{fancy}
\rhead{\today}
\lhead{\namn, \mail}
\chead{}
\lfoot{}
\cfoot{}
\rfoot{}

\cleardoublepage
\newpage
\dosecttoc 
\tableofcontents
\cleardoublepage

\rfoot{\thepage}
\pagenumbering{arabic}

\section{Problemspecifikation}\label{sec:problemspecifikation}
Laborationen gick ut på att skriva ett jobbhanteringssystem för att
hantera långsam kryptering av loggmeddelanden. Krypteringsalgoritmen
\textit{xorcrypt()} är tillräckligt långsam för att det krävs
arbetsköer för loggmeddelanden som väntar på att krypteras.
Loggmeddelanden kommer in till programmet via \textit{fifon}.

Problemspecifikation finns i original på sidan:\\
\begin{footnotesize}
\verb!http://www.cs.umu.se/kurser/5DV004/VT09/labbar/lab3/index.html!
\end{footnotesize}

\section{Användarhandledning}\label{sec:anvandarhandledning}
Källkoden till programmet finns i katalogen
\verb!~dit06ajn/edu/5DV004/lab3! och kompileras genom att i en
kommandopromt från denna katalog köra följande kommandon:

\begin{verbatim}
salt:dit06ajn:~/edu/5DV004/lab2$ make
salt:dit06ajn:~/edu/5DV004/lab2$ make install
\end{verbatim}

Detta i sin tur kommer att kompilera de nödvändiga filerna i katalogen
\verb!src! och placera en körbar fil \verb!msh! i katalogen
\verb!bin!. Programmet körs från katalogen \verb!bin! enligt syntax:

\begin{footnotesize}
\begin{verbatim}
./mcryptlogger -C [number of computers] -B [number of buffers]
               -P [number of threads]
\end{verbatim}
\end{footnotesize}

Ovan kommando utan parameter startar programmet med angivet
\textit{-C} hur många datorer (trådar) som ska användas för att
kryptera meddelanden som kommer in, \textit{-B} antalet buffrar som
kommer att användas i köerna och \textit{-P} antalet trådar som kommer
att användas för att bevaka \textit{fifo}s och läsa in meddelanden
från dem för placering i köer.

\section{Systembeskrivning}\label{sec:systembeskrivning}
Systemet består i stort sett av två köer och ett antal trådar som
hanterar olika flöden i programmets gång. Det finns en kö som håller
ofyllda buffrar som kommer att fyllas med meddelanden som läses in.
Den andra kön i programmet fylls med buffrar som redan är fyllda med
information från \textit{fifo}s. Två trådar med olika beteenden kommer
att startas upp, nedan följer kort beskrivning över dessa.

\subsection{Lästråd}\label{sec:lastrad}
Varje lästråd kommer att skapa en \textit{fifo} namngiven 1, 2, ...
[antalet trådar]. Dessa \textit{fifos} öppnas för läsning, när det
finns information så läses en byte och sedan väntar tråden på att det
ska finnas en ledig buffert i kön som håller dessa. När detta finns så
kommer denna buffert att fyllas och sättas in i kön som håller fyllda
buffrar.

\subsection{Krypteringstråd}\label{sec:krypteringstrad}
Varje krypteringstråd ställer sig och väntar på att det kommer in en
fylld buffert i kön som håller dessa. När detta inträffar kommer
meddelandet i bufferten att krypteras och den krypterade informationen
kommer att sparas i en fil med namn som börjar med vilken
\textit{fifo} meddelandet är inläst från avslutat med vilket
meddelande i turordningen det är, exempel \textit{0\_3} är tredje
meddelande från \textit{fifo} noll.

\subsection{Trådsäkerhet}\label{sec:tradsakerhet}
Eftersom det är flera trådar som jobbar parallellt i detta program
behövs det en del koller så att trådarna inte ändrar i samma buffrar
och variabler samtidigt. Den största delen av trådsäkerheten är
implementerad i köns \textit{dequeue}– och \textit{enqueue}–metoder.
Varje kö har en \textit{pthread\_mutex\_t} variabel och en
\textit{pthread\_cond\_t} variabel. När buffrar läggs in eller tas ut ur
kön så låses köns \textit{mutex} variabel. Detta innebär att det inte
kommer att bli några fel på grund av att två trådar anropar
\textit{dequeue} och \textit{enqueue} samtidigt.

Vid uttagande ur kön används dessutom \textit{pthread\_cond\_t}
variabeln för att kolla så att kön inte är tom. Så länge kön är tom
ställer sig tråden som anropar \textit{dequeue} och väntar på
\textit{pthread\_cond\_t}. När ett meddelande sätts in, via en annans
tråds anrop till \textit{enqueue}, körs metoden
\textit{pthread\_cond\_signal()} för att meddela en av väntande köer
att kön innehåller information. Dessa köer vaknar då upp och kollar om
det finns något att plocka ut, finns inte detta (någon annan tråd han
före) så ställer sig tråden återigen och väntar på
\textit{pthread\_cond\_t} variabeln.

\section{Testkörningar}\label{sec:testkorningar}
Följande utskrifter visar någrascenarion där spårutskrifter inte är
borttagna ur programmet:

Starta mcryptlogger med en krypteringstråd, 2 buffrar, och 2 lästrådar (\textit{fifos}):
\begin{footnotesize}
\begin{verbatim}
$:./mcryptlogger -C 1 -B 2 -P 2
filledbufqueue: front(0), count(0)
enqueue: waiting for lock!, 0
enqueue: locked!, 0
dequeue: unlocked!, 1
Queue: not empty! Cond signal, 1
enqueueing0
enqueue: waiting for lock!, 1
enqueue: locked!, 1
dequeue: unlocked!, 2
enqueueing1
freeBufQueue: front(0), count(2)
key: Detta är nyckel nr 0 på max 127 tecken ey.

Hello from readthread 0
filledBufQueue from readThread: front(0), count(0)
Hello from readthread 1
filledBufQueue from readThread: front(0), count(0)
Hello cryptThreadInit! 0
dequeue: waiting for lock!, 0
dequeue: locked!, 0
dequeue: Queue is empty, waiting for cond!
\end{verbatim}
\end{footnotesize}

Skicka 4000 bytes till \textit{fifo} 0, med hjälp av skriptet i bilaga
\ref{msg.sh}:
\begin{footnotesize}
\begin{verbatim}
./msg.sh 0
4+0 records in
4+0 records out
4000 bytes transferred in 0.000081 secs (49344753 bytes/sec)
\end{verbatim}
\end{footnotesize}

Utskrift från ovan kommando:
\begin{footnotesize}
\begin{verbatim}
dequeue: waiting for lock!, 2
dequeue: locked!, 2
Queue: dequeueing!, 2
dequeue: unlocked!, 1
ReadThread: Got a freeBuf.dequeue
front(0), count(0)
ReadThread: enqueuing new message
front(0), count(0)
enqueue: waiting for lock!, 0
enqueue: locked!, 0
dequeue: unlocked!, 1
Queue: not empty! Cond signal, 1
Queue: dequeueing!, 1
dequeue: unlocked!, 0
CryptThread: got a filledBufQueue.dequeue
message to crypt fifo: 0
kY\311\345xYd
fifo: 0, fifo_count: 0
dequeue: waiting for lock!, 0
dequeue: locked!, 0
dequeue: Queue is empty, waiting for cond!
\end{verbatim}
\end{footnotesize}

\section{Lösningen begränsningar}\label{sec:losningensbegransningar}
Implementationen så som den är inlämnad just när saknar hantering av
signaler på ett vettigt sätt, det hade behövts utförligare
testkörningar för att verifiera att programmet agerar som tänkt och
att det är stabilt då det verkar vara tänkt för att köras som en
\textit{demon}-process. Lösningens olika delar såsom kön och trådarna
hade behövt tester för att säkerställa att de fungerar som de ska.

\section{Problem och reflektioner}\label{sec:problemochreflektioner}
Eftersom jag fortfarande känner mig väldigt oerfaren av C känns det
som att man hade behövt en grundlig genomgång om \textit{heapen} och
\textit{stacken} så man vet vad som är vettigt att returnera från
funktioner så att man inte helt plötsligt råkar skriva över någon
minnesplats som fortfarande används av någon buffert.

\newpage
\appendix
\pagenumbering{arabic}
\section{Bilagor}\label{Bilagor}
% Källkoden ska finnas tillgänglig i er hemkatalog
% ~/edu/apjava/lab1/. Bifoga även utskriven källkod.
Härefter följer utskrifter från källkoden och andra filer som hör till
denna laboration.

\newpage
\subsection{Makefile}\label{Makefile}
\begin{scriptsize}
  \verbatiminput{../src/Makefile}
\end{scriptsize}

\subsection{mcryptlogger.c}\label{mcryptlogger.c}
\begin{scriptsize}
  \verbatiminput{../src/mcryptlogger.c}
\end{scriptsize}

\subsection{mcryptlogger.h}\label{mcryptlogger.h}
\begin{scriptsize}
  \verbatiminput{../src/mcryptlogger.h}
\end{scriptsize}

\subsection{msg.sh}\label{msg.sh}
\begin{scriptsize}
  \verbatiminput{../src/msg.sh}
\end{scriptsize}

\subsection{queue.c}\label{queue.c}
\begin{scriptsize}
  \verbatiminput{../src/queue.c}
\end{scriptsize}

\subsection{queue.h}\label{queue.h}
\begin{scriptsize}
  \verbatiminput{../src/queue.h}
\end{scriptsize}

\end{document}
